\begin{tabular}{lc}
\hline\hline 
\addlinespace 
Policy Options & Number of Meetings \\ 
\hline 
E & 3 \\
T & 7 \\
U & 1 \\
E,U & 31 \\
U,T & 43 \\
E,U,T & 49 \\
\addlinespace 
Total & 134 \\
\hline 
\end{tabular}